\chapter{Future Implementation}
When implementing this in the future there are a lot of things that should come into consideration. First off, the system architecture. We want the system to work as optimal as possible. Meaning that we would want the faster algorithms at the forefront doing the heavy load of the work before passing the results onto the slower algorithms. 

We should make sure that the algorithms implemented have various focus areas, so that we can cover multiple angles to doing text comparison (as seen in figure ~\ref{MultipleAlgo}). It should also be priority that algorithms only do a single task, and not try to do a full range of comparisons single handedly.

A lot of effort should also go into the preparation phase, where text is being normalized. We saw several examples of text normalization giving an inaccurate result, albeit only minor inaccuracies. However this error source might as well be eliminated to help get as an accurate score as possible.

Another thing to consider is the cost. A lot of time and manpower could be sunk into a task that would provide little value. There is a need to weigh cost and benefits for all extra implementation, but this is a talk for another time.

And finally a lot of testing of the score evaluation should be made. This part could end up being a total different method than what is currently is, by having many algorithms doing their separate scoring, but a lot of testing of this is still needed.