\chapter{General - Terms and Rules}
As there is a lot of terms used in this thesis, a short introduction to the most used are in order.
\begin{itemize}
\item \textbf{Article:} For this thesis, a digital document containing the contents of a piece of news. Could originate from papers, magazines, TV or other forms of media. For this thesis a document corresponds to an article. Articles are in their electronic form stored at Infomedia as XML files, I will throughout this thesis only deal with the part of the XML files that contains data of value to me in this assignment. This being \textit{Tags} (see below), \textit{Headline}, \textit{Sub headline} and \textit{Article Text}.
\item \textbf{Corpus:} From Latin meaning \textit{body}. In this thesis that describes the test set of articles being used a test set throughout my thesis.
\item \textbf{Tag:} Used in Ontology\footnote{\url{http://en.wikipedia.org/wiki/Ontology_(information_science)}} to create words that describes the contents of an article.
\item \textbf{Term:} Basically a word. A term will be something that can be searched for.
\end{itemize}

Hvornår er en artikel et duplikat (korte artikler (breaking news), hvornår er to artikler "tilstrækkeligt" forskellige?)
blah about copyright rules in Denmark...